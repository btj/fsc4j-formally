\documentclass{article}

\usepackage{amssymb}
\usepackage{mathpartir}

\title{FSC4J, Formally}

\author{Bart Jacobs}

\newcommand{\op}{\odot}
\newcommand{\hstep}{\rightarrow_\mathsf{h}}
\newtheorem{lemma}{Lemma}
\newtheorem{definition}{Definition}

\begin{document}

\maketitle

\section{MicroJava}

$$\begin{array}{r l}
\tau ::= & \mathsf{int}\ |\ \mathsf{bool}\ |\ C\ |\ \mathsf{void}\\
v ::= & z\ |\ o\ |\ \mathsf{null}\ |\ \mathsf{true}\ |\ \mathsf{false}\\
e ::= & v\ |\ x\ |\ e \op e\\
c ::= & e\ |\ \mathbf{new}\ C(\overline{e})\ |\ e.f\ |\ e.f := e\\
| & e.m(\overline{e})\ |\ c : \tau\ |\ \mathbf{let}\ x = c\ \mathbf{in}\ c\\
\mathit{meth} ::= & \tau\ m(\overline{\tau\ x})\ \{\ c\ \}\\
\mathit{class} ::= & \mathbf{class}\ C\ \{\ \overline{\tau\ f}\ C(\overline{\tau\ x})\ \{\ c\ \}\ \overline{\mathit{meth}}\ \}\\
\mathit{prog} ::= & \overline{\mathit{class}}\ c
\end{array}$$

Every value is of type $\mathsf{void}$; its default value is $\mathsf{null}$.

\begin{mathpar}
\inferrule{
o \notin \mathrm{dom}\,h\\
o : C\\
\overline{v} : \overline{\tau}\\
\mathbf{class}\ C\ \{\ C(\overline{\tau\ x})\ \{\ c\ \}\ \cdots\ \}\\
(h[o:=\{\overline{f \mapsto v_0^\tau}\}], c[o/\mathsf{this},\overline{v}/\overline{x}]) \Downarrow (h', \_)
}{
(h, \mathbf{new}\ C(\overline{v})) \Downarrow (h', o)
}
\and
\inferrule{
o \in \mathrm{dom}\,h\\
f \in \mathrm{dom}\,h(o)
}{
(h, o.f) \Downarrow (h, h(o)(f))
}
\and
\inferrule{
o \in \mathrm{dom}\,h\\
o : C\\
v : C.f
}{
(h, o.f := v) \Downarrow (h[o:=h(o)[f:=v]], v)
}
\and
\inferrule{
o \in \mathrm{dom}\,h\\
o : C\\
\mathbf{class}\ C\ \{\ \cdots\ \tau\ m(\overline{\tau\ x})\ \{\ c\ \}\ \cdots\ \}\\
(h, c[o/\mathsf{this}, \overline{v}/\overline{x}] : \tau) \Downarrow (h', v)
}{
(h, o.m(\overline{v})) \Downarrow (h', v)
}
\and
\inferrule{
(h, c) \Downarrow (h', v)\\
(h', c'[v/x]) \Downarrow (h'', v')
}{
(h, \mathbf{let}\ x = c\ \mathbf{in}\ c') \Downarrow (h'', v')
}
\and
\inferrule{
(h, c) \Downarrow (h', v)\\
v : \tau
}{
(h, c : \tau) \Downarrow (h', v)
}
\end{mathpar}

\section{Preconditions, postconditions, mutates, old}

We make the following simplifications with respect to real FSC4J: exactly one precondition, exactly one $\mathbf{mutates}$ clause which evaluates to a set of object references, exactly one postcondition, $\mathbf{old}$ expressions are desugared to a $\mathbf{let\_old}$ clause that binds a value computed in the pre-state to a variable available in the postcondition.

Notice that the $\mathbf{let\_old}$ expression can allocate objects; these remain in subsequent heaps, but outside of the postcondition they are not reachable. Notice also that there is exactly one $\mathbf{let\_old}$ expression; if multiple $\mathbf{old}$ expressions appear in the postcondition, they can be tupled up into a single object.

$$\begin{array}{r l}
\tau ::= & \mathsf{int}\ |\ \mathsf{bool}\ |\ C\ |\ \mathsf{set}\ |\ \mathsf{void}\\
v ::= & z\ |\ o\ |\ \mathsf{null}\ |\ \mathsf{true}\ |\ \mathsf{false}\ |\ \{\overline{v}\}\\
e ::= & v\ |\ x\ |\ e \op e\\
c ::= & e\ |\ \mathbf{new}\ C(\overline{e})\ |\ e.f\ |\ e.f := e\\
| & e.m(\overline{e})\ |\ c : \tau\ |\ \mathbf{let}\ x = c\ \mathbf{in}\ c\\
\mathit{spec} ::= & \mathbf{pre}\ c\ \mathbf{mutates}\ c\ \mathbf{let\_old}\ x = c\ \mathbf{post}\ c\\
\mathit{meth} ::= & \tau\ m(\overline{\tau\ x})\ \mathit{spec}\ \{\ c\ \}\\
\mathit{class} ::= & \mathbf{class}\ C\ \{\ \overline{\tau\ f}\ C(\overline{\tau\ x})\ \mathit{spec}\ \{\ c\ \}\ \overline{\mathit{meth}}\ \}\\
\mathit{prog} ::= & \overline{\mathit{class}}\ c
\end{array}$$

We instrument the semantics with a set $F$ of \emph{frozen objects}.

\begin{mathpar}
\inferrule{
o \notin \mathrm{dom}\,h\\
o : C\\
\overline{v} : \overline{\tau}\\
\mathbf{class}\ C\ \{\ C(\overline{\tau\ x})\ \mathbf{pre}\ c_\mathsf{P}\ \mathbf{mutates}\ c_\mathsf{M}\ \mathbf{let\_old}\ y = c_\mathsf{L}\ \mathbf{post}\ c_\mathsf{Q}\ \{\ c\ \}\ \cdots\ \}\\
(h, c_\mathsf{P}[\overline{v}/\overline{x}]) \Downarrow_{\mathrm{dom}\,h} (\_, \mathsf{true})\\
(h, c_\mathsf{M}[\overline{v}/\overline{x}]) \Downarrow_{\mathrm{dom}\,h} (\_, O)\\
O \subseteq \mathrm{dom}\,h\\
O \cap F = \emptyset\\
(h, c_\mathsf{L}[\overline{v}/\overline{x}]) \Downarrow_{\mathrm{dom}\,h} (h', v_\mathsf{L})\\
(h'[o:=\{\overline{f \mapsto v_0^\tau}\}], c[o/\mathsf{this},\overline{v}/\overline{x}]) \Downarrow_{\mathrm{dom}\,h' \setminus O \setminus \{o\}} (h'', \_)\\
(h'', c_\mathsf{Q}[o/\mathsf{this}, \overline{v}/\overline{x}, v_\mathsf{L}/y]) \Downarrow_{\mathrm{dom}\,h''} (\_, \mathsf{true})
}{
(h, \mathbf{new}\ C(\overline{v})) \Downarrow_F (h'', o)
}
\and
\inferrule{
o \in \mathrm{dom}\,h\\
f \in \mathrm{dom}\,h(o)
}{
(h, o.f) \Downarrow_F (h, h(o)(f))
}
\and
\inferrule{
o \in \mathrm{dom}\,h\\
o : C\\
v : C.f\\
o \notin F\\
}{
(h, o.f := v) \Downarrow_F (h[o:=h(o)[f:=v]], v)
}
\and
\inferrule{
o \in \mathrm{dom}\,h\\
o : C\\
\mathbf{class}\ C\ \{\ \cdots\ \tau\ m(\overline{\tau\ x})\ \mathbf{pre}\ c_\mathsf{P}\ \mathbf{mutates}\ c_\mathsf{M}\ \mathbf{let\_old}\ y = c_\mathsf{L}\ \mathbf{post}\ c_\mathsf{Q}\ \{\ c\ \}\ \cdots\ \}\\
(h, c_\mathsf{P}[o/\mathsf{this}, \overline{v}/\overline{x}]) \Downarrow_{\mathrm{dom}\,h} (\_, \mathsf{true})\\
(h, c_\mathsf{M}[o/\mathsf{this}, \overline{v}/\overline{x}]) \Downarrow_{\mathrm{dom}\,h} (\_, O)\\
O \subseteq \mathrm{dom}\,h\\
O \cap F = \emptyset\\
(h, c_\mathsf{L}[o/\mathsf{this}, \overline{v}/\overline{x}]) \Downarrow_{\mathrm{dom}\,h} (h', v_\mathsf{L})\\
(h', c[o/\mathsf{this}, \overline{v}/\overline{x}] : \tau) \Downarrow_{\mathrm{dom}\,h' \setminus O} (h'', v)\\
(h'', c_\mathsf{Q}[o/\mathsf{this}, \overline{v}/\overline{x}, v_\mathsf{L}/y, v/\mathsf{result}]) \Downarrow_{\mathrm{dom}\,h''} (\_, \mathsf{true})
}{
(h, o.m(\overline{v})) \Downarrow_F (h'', v)
}
\and
\inferrule{
(h, c) \Downarrow_F (h', v)\\
(h', c'[v/x]) \Downarrow_F (h'', v')
}{
(h, \mathbf{let}\ x = c\ \mathbf{in}\ c') \Downarrow_F (h'', v')
}
\and
\inferrule{
(h, c) \Downarrow_F (h', v)\\
v : \tau
}{
(h, c : \tau) \Downarrow_F (h', v)
}
\end{mathpar}

\begin{lemma}
Apart from allocating additional (unreachable) objects, the FSC4J semantics matches the MicroJava semantics:
if $(\emptyset, c) \Downarrow_\emptyset (h, v)$ then $\exists h' \subseteq h.\;(\emptyset, c) \Downarrow (h', v)$.
\end{lemma}

\begin{definition}
We say a heap $h$ is \emph{well-typed} in the context of a given program $\mathit{prog}$, denoted $\mathit{prog} \vdash h\,\mathsf{wt}$, if each allocated object's class appears in the program and it has exactly the fields declared by the class and their values are of the declared types:
$$\forall o \in \mathrm{dom},h.\;\exists C, \overline{\tau}, \overline{f}.\;o : C \land \mathbf{class}\ C\ \{\ \overline{\tau f}\ \cdots\ \} \land h(o) \in \{\{\overline{f\mapsto v}\}\ |\ \overline{v : \tau}\}$$
\end{definition}

\begin{definition}
Given a program, we say a method $C.m$'s specification is \emph{well-defined} if the clauses always evaluate to a well-typed value:
$$\begin{array}{l}
\mathsf{spec\_well\_defined}(C.m) \triangleq\\
\quad \forall h, \overline{v}.\;h\,\mathsf{wt} \land o : C \land o \in \mathrm{dom},h \land \overline{v : \tau} \land \{\overline{v}\} \cap \mathcal{O} \subseteq \mathrm{dom}\,h \Rightarrow\\
\quad\quad \exists v_\mathsf{P}.\;(h, c_\mathsf{P}[o/\mathsf{this},\overline{v}/\overline{x}]) \Downarrow_{\mathrm{dom}\,h} (\_, v_\mathsf{P}) \land v_\mathsf{P} : \mathsf{bool} \land
(v_\mathsf{P} = \mathsf{true} \Rightarrow\\
\quad\quad\quad \exists O, h', v_\mathsf{L}.\;(h, c_\mathsf{M}[o/\mathsf{this},\overline{v}/\overline{x}]) \Downarrow_{\mathrm{dom}\,h} (\_, O) \land O \subseteq \mathrm{dom}\,h\; \land\\
\quad\quad\quad\quad (h, c_\mathsf{L}[o/\mathsf{this},\overline{v}/\overline{x}]) \Downarrow_{\mathrm{dom}\,h} (h', v_\mathsf{L})\; \land\\
\quad\quad\quad\quad \forall h'', v.\;h''\,\mathsf{wt} \land \mathrm{dom}\,h' \subseteq \mathrm{dom}\,h'' \land h'|_{\mathrm{dom}\,h'\setminus O} = h''|_{\mathrm{dom}\,h'\setminus O} \land v : \tau \land \{v\} \cap \mathcal{O} \subseteq \mathrm{dom}\,h'' \Rightarrow\\
\quad\quad\quad\quad\quad \exists v_\mathsf{Q}.\;(h'', c_\mathsf{Q}[o/\mathsf{this}, \overline{v}/\overline{x}, v_\mathsf{L}/y, v/\mathsf{result}]) \Downarrow_{\mathrm{dom}\,h''} (\_, v_\mathsf{L}) \land v_\mathsf{L} : \mathsf{bool})\\
\textrm{where $\mathbf{class}\ C\ \{\ \cdots\ \tau\ m(\overline{\tau\ x})\ \mathbf{pre}\ c_\mathsf{P}\ \mathbf{mutates}\ c_\mathsf{M}\ \mathbf{let\_old}\ y = c_\mathsf{L}\ \mathbf{post}\ c_\mathsf{Q}\ \{\ c\ \}$}
\end{array}$$
\end{definition}

\begin{definition}
Given a program, a method with a well-defined specification \emph{satisfies its specification} if, for any well-typed call of the method and well-typed pre-heap in which the precondition evaluates to $\mathsf{true}$, the body evaluates to a value and the postcondition evaluates to $\mathsf{true}$:
$$\begin{array}{l}
\mathsf{satisfies\_spec}(C.m) \triangleq\\
\quad \forall h, o, \overline{v}.\;h\,\mathsf{wt} \land \{o, \overline{v}\} \cap \mathcal{O} \in \mathrm{dom}\,h \land \overline{v} : \overline{\tau} \land (h, c_\mathsf{P}[o/\mathsf{this},\overline{v}/\overline{x}]) \Downarrow_{\mathrm{dom}\,h} (\_, \mathsf{true}) \Rightarrow\\
\quad \exists O, h', v_\mathsf{L}, h'', v.\;(h, c_\mathsf{M}[o/\mathsf{this},\overline{v}/\overline{x}]) \Downarrow_{\mathrm{dom}\,h} (\_, O) \land (h, c_\mathsf{L}[o/\mathsf{this},\overline{v}/\overline{x}]) \Downarrow_{\mathrm{dom}\,h} (\_, v_\mathsf{L})\;\land\\
\quad\quad (h', c[o/\mathsf{this},\overline{v}/\overline{x}]) \Downarrow_{\mathrm{dom}\,h\setminus O} (h'', v) \land v : \tau\;\land\\
\quad\quad (h'', c_\mathsf{Q}[o/\mathsf{this},\overline{v}/\overline{x}, v_\mathsf{L}/y, v/\mathsf{result}]) \Downarrow_{\mathsf{dom}\,h''} (\_, \mathsf{true})\\
\textrm{where $\mathbf{class}\ C\ \{\ \cdots\ \tau\ m(\overline{\tau\ x})\ \mathbf{pre}\ c_\mathsf{P}\ \mathbf{mutates}\ c_\mathsf{M}\ \mathbf{let\_old}\ y = c_\mathsf{L}\ \mathbf{post}\ c_\mathsf{Q}\ \{\ c\ \}$}
\end{array}$$
\end{definition}

\section{Data abstraction}

The system from the previous section suffices for basic procedural abstraction, but does not support data abstraction. In particular, for effective data abstraction, it must be possible for clients to tell whether a mutation affects an inspector's value without needing to look at the inspector's implementation. To support this, we extend the system so that it tracks the set of objects that were read during the evaluation of a command and makes this set available in a method's postcondition under the name $\mathsf{readSet}$. 

Furthermore, to support checking abstract state invariants without introducing infinite recursion, we make the caller's frozen set available to a callee's precondition under the name $\mathsf{frozenSet}$. This allows the callee to determine whether a caller possibly mutated an object and hence possibly broke its invariants. This enables avoiding that inspector calls performed by inspectors cause the recursive checking of abstract state invariants.

These ingredients are sufficient to encode representation objects, representation invariants, and abstract state invariants. The pattern is as follows:
\begin{itemize}
\item Each class defines a method $\mathsf{repSet}()$ that returns the set of objects it uses for an instance $o$'s representation. In FSC4J, this corresponds to $o$ itself plus the union of the rep sets of $o$'s representation objects. Method $\mathsf{repSet}$ should have a postcondition asserting that its result value be a subset of its read set. In other words, to compute its rep set, an object must inspect only objects that are in its rep set. (In the literature, such an inspector is called \emph{self-framing}.)
\item Each class defines a method $\mathsf{repInv}()$ that returns a boolean indicating whether an instance's representation invariants hold. It shall first call the instance's representation objects' $\mathsf{repInv}$ methods. This method shall have a postcondition asserting that the read set is included in $o$'s rep set. That is, an object's validity must depend only on the state of the objects in its rep set.
\item Each class defines a method $\mathsf{absInv}()$ that returns a boolean indicating whether an instance's abstract state invariants hold. (We say a class has an \emph{invalid representation} if its representation invariant does not imply its abstract state invariant.)
\item Each method that requires that an object $o$ be valid shall require $o.\mathsf{repInv}()$, as well as, if $o$ is not in $\mathsf{frozenSet}$, $o.\mathsf{absInv}()$.
\item Each method that mutates an object $o$ and restores its validity shall ensure $o.\mathsf{repInv}()$ as well as $o.\mathsf{absInv}()$.
\item Each inspector's postcondition shall assert that its read set be included in the union of the rep sets of the objects inspected by the inspector.
\end{itemize}

A constructor or factory method's postcondition typically needs to express that the newly created object's rep set is included in the set of objects that were allocated during the evaluation of the constructor or method. To support this, we make the set of objects allocated by a method available to the method's postcondition under the name $\mathsf{newSet}$. Note: a constructor's $\mathsf{newSet}$ includes $\mathsf{this}$.

\subsection{Example}

Consider the following FSC4J classes:

\begin{verbatim}
/** @invar | 0 <= getValue() */
public class NonnegCell {

    /** @invar | 0 <= value */
    private int value;
    
    public int getValue() { return value; }
    
    /**
     * @pre | 0 <= value
     * @post | getValue() == value
     */
    public NonnegCell(int value) { this.value = value; }
    
    /**
     * @pre | 1 <= getValue()
     * @mutates | this
     * @post | getValue() == old(getValue()) - 1
     */
    public void decrement() { this.value--; }

}

/** @invar | 1 <= getValue() */
public class PosCell {
    
    /**
     * @invar | inner != null && 1 <= inner.getValue()
     * @representationObject
     */
    private NonnegCell inner;
    
    public int getValue() { return inner.getValue(); }
    
    /**
     * @pre | 1 <= value
     * @post | getValue() == value
     */
    public PosCell(int value) { inner = new NonnegCell(value); }
    
    /**
     * @pre | 2 <= getValue()
     * @mutates | this
     * @post | getValue() == old(getValue()) - 1
     */
    public void decrement() { inner.decrement(); }
    
}
\end{verbatim}

This translates to the following in the formal language. Note: we elide $\mathbf{pre}\ \mathsf{true}$, $\mathbf{mutates}\ \emptyset$, and $\mathbf{let\_old}\ \mathsf{dummy} = \mathsf{null}$. Also, we write $c \odot c'$ as a shorthand for $\mathbf{let}\ x = c\ \mathbf{in}\ \mathbf{let}\ x' = c'\ \mathbf{in}\ x \odot x'$.

$$\begin{array}{l}
\mathbf{class}\ \mathsf{NonnegCell}\ \{\\
\quad \mathsf{int}\ \mathsf{value};\\
\\
\quad \mathsf{set}\ \mathsf{repSet}()\\
\quad\quad \mathbf{post}\ \mathsf{readSet} \subseteq \mathsf{result}\\
\quad \{\ \{\mathsf{this}\}\ \}\\
\\
\quad \mathsf{bool}\ \mathsf{repInv}()\\
\quad\quad \mathbf{post}\ \mathsf{readSet} \subseteq \mathsf{this}.\mathsf{repSet}()\\
\quad \{\ 0 \le \mathsf{this}.\mathsf{value} \}\\
\\
\quad \mathsf{bool}\ \mathsf{absInv}()\\
\quad\quad \mathbf{post}\ \mathsf{true}\\
\quad \{\ 0 \le \mathsf{this}.\mathsf{getValue}() \}\\
\\
\quad \mathsf{int}\ \mathsf{getValue}()\\
\quad\quad \mathbf{pre}\ \mathsf{this}.\mathsf{repInv}() \land (\mathsf{this} \in \mathsf{frozenSet} \lor \mathsf{this}.\mathsf{absInv}())\\
\quad\quad \mathbf{post}\ \mathsf{readSet} \subseteq \mathsf{this}.\mathsf{repSet}()\\
\quad \{\ \mathsf{this}.\mathsf{value} \}\\
\\
\quad \mathsf{NonnegCell}(\mathsf{int}\ \mathsf{value})\\
\quad\quad \mathbf{pre}\ 0 \le \mathsf{value}\\
\quad\quad \mathbf{post}\ \mathsf{this}.\mathsf{getValue}() = \mathsf{value} \land \mathsf{this}.\mathsf{repSet}() \subseteq \mathsf{newSet}\\
\quad \{\ \mathsf{this}.\mathsf{value} := \mathsf{value}\ \}\\
\\
\quad \mathsf{void}\ \mathsf{decrement}()\\
\quad\quad \mathbf{pre}\ \mathsf{this}.\mathsf{repInv}() \land (\mathsf{this} \in \mathsf{frozenSet} \lor \mathsf{this}.\mathsf{absInv}()) \land 1 \le \mathsf{this}.\mathsf{getValue}()\\
\quad\quad \mathbf{mutates}\ \mathsf{this}.\mathsf{repSet}()\\
\quad\quad \mathbf{let\_old}\ \mathsf{oldValue} = \mathsf{this}.\mathsf{getValue}()\\
\quad\quad \mathbf{post}\ \mathsf{this}.\mathsf{repInv}() \land \mathsf{this}.\mathsf{absInv}() \land \mathsf{this}.\mathsf{getValue}() = \mathsf{oldValue} - 1\\
\quad \{\ \mathsf{this}.\mathsf{value} := \mathsf{this}.\mathsf{value} - 1\ \}\\
\}
\end{array}$$

$$\begin{array}{l}
\mathbf{class}\ \mathsf{PosCell}\ \{\\
\quad \mathsf{NonnegCell}\ \mathsf{inner};\\
\\
\quad \mathsf{set}\ \mathsf{repSet}()\\
\quad\quad \mathbf{post}\ \mathsf{readSet} \subseteq \mathsf{result}\\
\quad \{\ \{\mathsf{this}\} \cup \mathbf{if}\ \mathsf{this}.\mathsf{inner} = \mathsf{null}\ \mathbf{then}\ \emptyset\ \mathbf{else}\ \mathsf{this}.\mathsf{inner}.\mathsf{repSet}()\ \}\\
\\
\quad \mathsf{bool}\ \mathsf{repInv}()\\
\quad\quad \mathbf{post}\ \mathsf{readSet} \subseteq \mathsf{this}.\mathsf{repSet}()\\
\quad \{\ (\mathsf{this}.\mathsf{inner} = \mathsf{null} \lor \mathsf{this}.\mathsf{inner}.\mathsf{repInv}()) \land \mathsf{this}.\mathsf{inner} \neq \mathsf{null} \land 1 \le \mathsf{this}.\mathsf{inner}.\mathsf{getValue}()\ \}\\
\\
\quad \mathsf{bool}\ \mathsf{absInv}()\\
\quad\quad \mathbf{post}\ \mathsf{true}\\
\quad \{\ 1 \le \mathsf{this}.\mathsf{getValue}() \}\\
\\
\quad \mathsf{int}\ \mathsf{getValue}()\\
\quad\quad \mathbf{pre}\ \mathsf{this}.\mathsf{repInv}() \land (\mathsf{this} \in \mathsf{frozenSet} \lor \mathsf{this}.\mathsf{absInv}())\\
\quad\quad \mathbf{post}\ \mathsf{readSet} \subseteq \mathsf{this}.\mathsf{repSet}()\\
\quad \{\ \mathsf{this}.\mathsf{inner}.\mathsf{getValue}() \}\\
\\
\quad \mathsf{PosCell}(\mathsf{int}\ \mathsf{value})\\
\quad\quad \mathbf{pre}\ 1 \le \mathsf{value}\\
\quad\quad \mathbf{post}\ \mathsf{this}.\mathsf{getValue}() = \mathsf{value} \land \mathsf{this}.\mathsf{repSet}() \subseteq \mathsf{newSet}\\
\quad \{\ \mathsf{this}.\mathsf{inner} := \mathbf{new}\ \mathsf{NonnegCell}(\mathsf{value})\ \}\\
\\
\quad \mathsf{void}\ \mathsf{decrement}()\\
\quad\quad \mathbf{pre}\ \mathsf{this}.\mathsf{repInv}() \land (\mathsf{this} \in \mathsf{frozenSet} \lor \mathsf{this}.\mathsf{absInv}()) \land 2 \le \mathsf{this}.\mathsf{getValue}()\\
\quad\quad \mathbf{mutates}\ \mathsf{this}.\mathsf{repSet}()\\
\quad\quad \mathbf{let\_old}\ \mathsf{oldValue} = \mathsf{this}.\mathsf{getValue}()\\
\quad\quad \mathbf{post}\ \mathsf{this}.\mathsf{repInv}() \land \mathsf{this}.\mathsf{absInv}() \land \mathsf{this}.\mathsf{getValue}() = \mathsf{oldValue} - 1\\
\quad \{\ \mathsf{this}.\mathsf{inner}.\mathsf{decrement}()\ \}\\
\}
\end{array}$$

\section{Immutable objects}

To facilitate reasoning about immutable objects, we extend the system to support \emph{permanently freezing} an object, as well as determining whether an object has been permanently frozen. Reading a permanently frozen object does not add the object to the read set. An immutable class' inspectors' postconditions shall specify that their read sets are empty.

\end{document}
